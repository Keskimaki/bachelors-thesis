\chapter{Johdanto\label{intro}}

Konttiteknologiaa käytetään kasvavissa määrissä monilla ohjelmistotuotannon osa alueilla. Konttiteknologia mahdollistaa fyysisen palvelimen tai ohjelmistökehittäjän tietokoneen suoritusympäristöstä eroavan vakaan ja rajattomasti toistettavan ympäristön. Konttiteknologian kasvava käyttö on luonut tarpeen hallinnoida suuria määriä kontteja samanaikaisesti. Tähän tarpeeseen pyrkivät vastaamaan erilaiset konttiorkestraatioalustat. \cite{Watada19}

Konttiorkestraatioalustat hallinnoivat suuria määriä kontteja jaetussa suoritusympäristössä. Orkestraatioalustat, kuten Kubernetes, mahdollistavat muun muassa palveluiden replikoinnin, tehokkaan julkaisuketjun, kontitettujen palveluiden monitoroinnin ja virhetiloista toipumisen. \cite{Khan17}

Konttien orkestrointi täydentää DevOps-toimintamallia, joka pyrkii jatkuvaan integraatioon ja nopeaan julkaisusykliin. DevOps-toimintamalli perustuu automaattisiin prosesseihin, jotka sallivat nopeamman muutosten viennin tuotantoon. Konttiteknologia mahdollistaa saman kontitetun ympäristön käyttämisen aina kehitysvaiheesta, testaukseen ja tuotantokäyttöön. \cite{Jabbari16}
