\chapter{Johdanto\label{intro}}

Konttiteknologiaa käytetään laajalti useilla ohjelmistotuotannon osa-alueilla.
Konttiteknologia mahdollistaa palvelimen suoritusympäristöstä erillisen vakaan ja toistettavan ympäristön \cite{Watada19}.
Konttiorkestrointialustat hallinnoivat suuria määriä kontteja jaetussa klusterissa.
Konttiorkestrointialustat, kuten Kubernetes, mahdollistavat muun muassa palveluiden replikoinnin, tehokkaan julkaisuketjun, kontitettujen palveluiden monitoroinnin ja virhetiloista toipumisen \cite{Khan17}.

Konttiteknologiaa ja konttien orkestrointia käytetään usein osana DevOps-toimintamalliin perustuvaa ohjelmistotuotantoa \cite{Kang16, Narasimhulu23}.
DevOps-toimintamalli pyrkii jatkuvaan integraatioon ja nopeaan julkaisusykliin.
Toimintamalli pohjautuu julkaisun ja laadunvalvonnann osalta tarkasti määriteltyihin automaattisiin prosesseihin \cite{Jabbari16}.

Tämä tutkielma tarkastelee DevOps:ia käsitteenä sekä konttiteknologian ja konttien orkestroinnin käyttöä osana DevOps-toimintamallia.
Näin tarkoituksena on todeta konttien orkestroinnin ja DevOps-toimintamallin välinen yhteensopivuus.
Yhteensopivuutta arvioidaan määrittelemällä DevOps-toimintamalliin kuuluvia vaiheita ja toimintatapoja sekä arvioimalla konttiteknologian ja konttien orkestroinnin tarjoamia etuja näiden toteuttamiseen.

Konttiteknologian ja konttien orkestroinnin lisäksi käsitellään myös muita mahdollisia ratkaisuja sekä niiden vaikutusta DevOps-toimintamallin mukaiseen ohjelmistotuotantoon.
Lopuksi esitetään käytännön esimerkki tutkielmassa käsiteltyjen teknologioiden ja DevOps-toimintamallin käytöstä.

Luvussa \ref{devops} annetaan määritelmä termille DevOps ja erotellaan siitä tutkielman kannalta merkitykselliset osa-alueet.
Tämän lisäksi kuvataan DevOps-toimintamalliin usein kuuluvia toimintatapoja sekä toimintamallin etuja ja siihen liittyviä haasteita.
Luvussa \ref{orchestration} käsitellään konttiteknologiaa ja konttien orkestrointia sekä niiden sopivuutta osaksi DevOps-toimintamallin mukaista ohjelmistotuotantoa.
Konttien orkestrointia käsitellään erityisesti Kubernetes-konttiorkestrointialustan kautta.

Luvussa \ref{options} käsitellään muita mahdollisia julkaisutapoja ja orkestrointiratkaisuja sekä niiden käyttöä DevOps-toimintamallin kanssa.
Konttiteknologian lisäksi mahdollisina julkaisutapoina käsitellään virtuaalikoneita ja suoraan fyysiselle palvelimelle julkaisemista.
Orkestrointiratkaisuna käsitellään konttien orkestroinnin lisäksi pilvialustojen käyttöä sekä konttien orkestroinnin ja pilvialustan käytön yhdistämistä.
Uutena konttiteknologiaa käyttävänä ratkaisuna mainitaan myös palvelimeton arkkitehtuuri.
Luvussa \ref{example} esitetään Helsingin yliopiston Norppa-palautejärjestelmä käytännön esimerkkinä DevOps-toimintamallin ja konttien orkestroinnin käytöstä.
Luvussa kuvataan järjestelmän perustoiminnot, julkaisuputki ja ohjelmistoarkkitehtuuri DevOps-toimintamallin ja konttien orkestroinnin käytön näkökulmasta.
