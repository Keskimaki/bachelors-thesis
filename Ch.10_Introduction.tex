\chapter{Johdanto\label{intro}}

% Rewrite all of this later

Konttiteknologiaa käytetään kasvavissa määrissä monilla ohjelmistotuotannon osa alueilla.
Konttiteknologia mahdollistaa fyysisen palvelimen tai ohjelmistökehittäjän tietokoneen suoritusympäristöstä eroavan vakaan ja rajattomasti toistettavan ympäristön \cite{Watada19}.
Konttiteknologian kasvava käyttö on luonut tarpeen hallinnoida suuria määriä kontteja samanaikaisesti.
Tähän tarpeeseen pyrkivät vastaamaan erilaiset konttiorkestraatioalustat.

Konttiorkestraatioalustat hallinnoivat suuria määriä kontteja jaetussa klusterissa.
Orkestraatioalustat, kuten Kubernetes, mahdollistavat muun muassa palveluiden replikoinnin, tehokkaan julkaisuketjun, kontitettujen palveluiden monitoroinnin ja virhetiloista toipumisen \cite{Khan17}.

Konttiteknologiaa ja konttien orkestraatiota käytetään usein osana DevOps-toimintamalliin perustuvaa ohjelmistotuotantoa \cite{Kang16}.
DevOps-toimintamalli pyrkii jatkuvaan integraatioon ja nopeaan julkaisusykliin.
Toimintamalli pohjautuu julkaisun ja laadunvalvonnann osalta tarkasti määriteltyihin automaattisiin prosesseihin \cite{Jabbari16}.

Tämän tutkielman tavoitteena on tarkastella DevOps:ia käsitteenä ja konttiteknologian sekä konttien orkestraation käyttöä osana DevOps-toimintamallia.
Näin tarkoituksena on todeta konttien orkestroinnin ja DevOps-toimintamallin välinen yhteensopivuus.
Konttien orkestroinnin lisäksi käsitellään myös muita mahdollisia ratkaisuja ja niiden vaikutusta DevOps-toimintamallin mukaiseen ohjelmistotuotantoon.

% Rewrite once you actually know what is going on
Luvussa \ref{devops} annetaan määritelmä DevOps-termille ja erotellaan siitä tutkielman kannalta merkitykselliset osa-alueet.
Tämän jälkeen kuvataan DevOps-toimintamalliin usein kuuluvia toimintatapoja.
Luvussa \ref{orchestration} käsitellään konttiteknologiaa ja konttien orkestrointia. Orkestraatiota käsitellään Kubernetes-konttiorkestraatioalustan kautta.
