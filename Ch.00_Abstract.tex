\begin{abstract}

DevOps pyrkii yhdistämään ohjelmistokehityksen ja IT-toiminnot.
DevOps-toimintamallin mukainen ohjelmistokehitys mahdollistaa automaattisen laadunvalvontaan ja nopean julkaisusyklin.
Toimintamallin mukaista ohjelmistokehitystä ja sen osana tapahtuvaa jatkuvaa integraatiota ja toimitusta voidaan tukea konttiteknologioiden ja konttien orkestroinnin avulla.
Konttiteknologian avulla samankaltaista vakaan ja toistettavan ympäristön tarjoavaa konttia voidaan käyttää eri vaiheissa toimintamallia.
Konttiorkestraatioalustat taas tarjoavat monia toimintallia tukevia palveluita testi- ja tuotantoympäristöissä. 

Tämän tutkielman tavoitteena on tarkastella DevOps:ia käsitteenä ja konttiteknologian
sekä konttien orkestraation käyttöä osana DevOps-toimintamallia. Näin tarkoituksena on
todeta konttien orkestroinnin ja DevOps-toimintamallin välinen yhteensopivuus. Konttien
orkestroinnin lisäksi käsitellään myös muita mahdollisia ratkaisuja ja niiden vaikutusta
DevOps-toimintamallin mukaiseen ohjelmistotuotantoon.
\end{abstract}
