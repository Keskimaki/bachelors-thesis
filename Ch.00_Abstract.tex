\begin{abstract}

DevOps tarkoittaa ohjelmistokehityksen ja IT-toimintojen yhdistämistä.
DevOps-toimintamalli mukainenen ohjelmistotuotanto pyrkii automaattiseen laadunvalvontaan ja nopeaan julkaisusykliin.
Toimintamallin mukaista ohjelmistotuotantoa ja sen osana tapahtuvaa jatkuvaa integraatiota ja toimitusta voidaan tukea konttiteknologioiden ja konttien orkestroinnin avulla.
Konttiteknologian avulla samankaltaista vakaan ja toistettavan ympäristön tarjoavaa konttia voidaan käyttää eri vaiheissa toimintamallia.
Konttiorkestraatioalustat taas tarjoavat toimintamallia tukevia palveluita testi- ja tuotantoympäristöissä.
Konttiteknologian sijaan julkaisutapana voi toimia myös virtuaalikone.
Orkestrointi voidaan toteuttaa konttien orkestroinnin lisäksi myös hyödyntämällä pilvialustoja.

Tämän tutkielma tarkastelee DevOps:ia käsitteenä ja konttiteknologian sekä konttien orkestraation käyttöä osana DevOps-toimintamallia. Näin tarkoituksena on todeta konttien orkestroinnin ja DevOps-toimintamallin välinen yhteensopivuus.
Konttien orkestroinnin lisäksi käsitellään myös muita mahdollisia ratkaisuja, kuten virtuaalikoneita ja pilvialustoja.
Tutkielmassa esitetään myös Helsingin yliopiston Norppa-palautejärjestelmä käytännön esimerkkinä konttiteknologian ja konttien orkestroinnin hyödyntämisestä DevOps-toimintamallin mukaisessa ohjelmistotuotannossa.

\end{abstract}
