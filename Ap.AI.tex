\appendix{Tekoälyn käyttö tutkielmassa\label{appendix:ai}}

Tutkielman suunnitteluvaiheessa käytettiin GPT-4-kielimallia OpenAI:n rajapinnan kautta.
Kielimallia käytettiin mahdollisten aiheiden generointiin ja aihealueen yleiseen tutkimiseen.
Kielimallia ei ole käytetty tutkielman kirjoituksen aloittamisen jälkeen ja mitään osaa tutkielmasta ei ole tuotettu kielimallin avulla.

DeepL-konekäännöspalvelua on käytetty suomenkielisten vastineiden löytämiseksi joillekin englanninkielisille termeille.
Tämän lisäksi palvelua on käytetty synonyymien etsimiseen.
Lähteiden hakuun, analysointiin tai tiivistämiseen ei ole käytetty tekoälypalveluita.
