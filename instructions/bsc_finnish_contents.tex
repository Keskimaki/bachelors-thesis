\chapter{Johdanto}


Kaikessa julkaistavaksi tarkoitetussa tekstissä kirjoittajan luomisen ja
esitystavan vapautta rajoittavat monet ohjeet ja tarkatkin määräykset.

Parhaimmillaan lukijalle ja kirjoittajalle yhteinen, tuttu säännöstö luo
eräänlaisen tukiverkoston, joka tukee sanoman siirtymistä vääristymättä.
Kirjoituksen lukija löytää kirjoituksesta helpommin olennaisen sisällön,
jos kirjoituksen ulkoasu ja sisällön rakenne vastaavat hänen
tottumuksiaan. Sama koskee myös kirjoittajaa. Noudattaessaan valmista
esitystapamallia kirjoittajan ei tarvitse käyttää aikaansa itse työn
kannalta toissijaisten seikkojen miettimiseen, vaan hän voi keskittyä
hiomaan tekstin sisältöä. Siksi kannattaa harjoitella myös työn ulkoasua
koskevien ohjeiden noudattamista, vaikka omasta mielestään osaisikin
valita esitykselleen ohjetta paremman muodon.

Tämä kirjoitus on tarkoitettu Helsingin yliopiston
Tietojenkäsittelytieteen osastoon alempien opinnäytteiden ja
harjoitusten ulkoasun ja rakenteen ohjeeksi. Ohje soveltuu siten
kandidaatintutkielman kirjoittamisen kurssille, ohjelmistotuotantoprojekteihin, seminaareihin ja
pro gradu -tutkielmiin. (Kirjoitus on päivitetty uusintapainos aiemmista
ohjeista, jotka kurssin luennoijat ovat laatineet \citep{erkio01,erkiomakela96,erkio94,verkamo92}.)

Tyylimäärittely on saatavissa pdflatex- ja word-versiona.
Tyylimäärittelyitä valitessa on huomattava ohjeet tekstien syöttöön
liittyvässtä koodauksesta (UTF8,ISO 8859-15).
Tämän kirjoituksen tukena sopivat käytettäväksi tavanomaiset latex- tai
word-oppaat. 

\chapter{Kirjoituksen rakenne}

Tarkastellaan aluksi tieteelliseltä tekstiltä odotettuja
kirjoituksen osia. Samoihin asioihin on luonnollisesti syytä
kiinnittää huomiota myös muussa teknisessä kirjoittamisessa. Huomattakoon, että tämä teksti ei ole tieteellinen teksti, eikä siten itse sisällä kaikkia niitä elementtejä, jotka tieteellisen tekstin sisällölliseen antiin kuuluvat. Tällaisia puutteita ovat esimerkiksi johdannon tutkimuskysymyksen asettelun puuttuminen sekä arvoivan materiaalin puute tekstin lopussa, sekä yhteenvedon latteus.
Teksti rajoittuu siten otsikkonsa mukaisesti vain tekniseen sisällön asetteluun.

\section{Tiivistelmä}



Tiivistelmäsivu sisältää seuraavat osat: työn bibliografiset tiedot,
tiivistelmäteksti, aiheluokat ja avainsanat. Bibliografiset tiedot
koostuvat työn otsikosta, tekijän nimestä, julkaisupaikan tiedoista,
julkaisuajankohdasta ja sivumäärästä.

Tiivistelmäteksti on lyhyt, yleensä yhden kappaleen mittainen
(maksimissaan noin 100 sanaa) selvitys
kirjoituksen tärkeimmästä sisällöstä: mitä on tutkittu, miten on
tutkittu ja mitä tuloksia on saatu.


Aiheluokat kuvataan ACM Computing Classification System -luokituksen (CCS)
luokituksen mukaisesti. Luokittelussa käytetään täysia polkuja juurisolmun CCS osoittamista lähtöposteistä lehtisolmuihin. Polkuja voi antaa 1-3 aihepiirien soveltuvuuden mukaan, mitä alempi opinnäyte, sen vähemmän polkuja se tarvitsee. 
Poluissa tasot erotetaan toisistaan nuolella eteenpäin. Kun  polun nimisanoja arvioidaan suhteessa työn sisältöön, merkitään boldface-fontilla tärkein termi, italics-fontilla toiseksi tärkein. Näin menetellään, mikäli jotkin termeistä ovat olennaisesti paremmin kuvaavia kuin muut polun termit. Nimettyjen polkujen lisäksi lukija voi siten tarkastella lisäulottuvuutena myös tärkeiksi merkittyjen termien joukkoa sinänsä.
Avainsanoiksi valitaan kirjoituksen sisältöä
keskeisesti kuvaavia käsitteitä.

\section{Johdanto}


Johdannon tarkoituksena on kertoa yleiskielisesti
työn tavoite. Kerrotaan (kuten tiivistelmässäkin, mutta laveammin),
mitä on tutkittu, miten on tutkittu ja mitä tuloksia on saatu.
Jotta kysymyksenasettelu ja tulokset on lukijan helppo oikein tulkita on syytä aloittaa johdanto asettelemalla tutkimus asiayhteyteensä, esimerkiksi kertomalla aluksi, minkälaisessa yhteydessä tarkasteluun otettavat haasteet esiintyvät ja keiden on ratkaisuista tarkoitus hyötyä.

Johdannon pituus määräytyy suhteessa koko kirjoitelman pituuteen.
Parisivuinen kirjoitus ei erikseen otsikoitua johdantoa kaipaa, sillä
se itsessään on laajennettu tiivistelmä. Kymmensivuisen
kirjoituksen johdanto voi olla vaikkapa sivun tai puolentoista
mittainen. Pro gradu -tutkielman 50-70-sivuiseen kokonaisuuteen
tuntuu 2-4-sivuinen johdanto kohtuulliselta. 

Johdanto kertoo siis lyhyessä, yleistajuisessa muodossa
koko kirjoitelman kysymyksenasettelun, juonen sekä tulokset ja johtopäätelmät.
Tämän luettuaan lukija voi päätellä, haluaako syventyä asiaan tarkemmin
lukemalla koko kirjoituksen.


\section{Käsittelyluvut}

Käsittelylukujen työnjako määräytyy käsiteltävän asian luonteen
mukaisesti.
Lukijan ohjailemiseksi kukin pääluku kannattaa aloittaa lyhyellä
kappaleella, joka paljastaa mikä kyseisen luvun keskeisin sisältö on ja
kuinka aliluvuissa asiaa kehitellään eteenpäin.
Erityisesti kannattaa kiinnittää huomiota siihen, että lukijalle ilmaistaan selkeästi miksi kutakin asiaa käsitellään ja miten käsiteltävät asiat suhtautuvat toisiinsa. 

Jäsentelyongelmista kielivät tilanteet, joissa
alilukuja on vain yksi, tai joissa käytetään useampaa kuin
kahta tasoa (pääluku ja sen aliluvut). Kolmitasoisia
otsikointeja saatetaan tarvita joissakin teknisissä
dokumenteissa perustellusti, mutta nämä muodostavat poikkeuksen.

Perusohjeena on käyttää tekstin rakenteellisesti painokkaita paikkoja,
kuten lukujen avauksia ja teksikappaleiden aloitusvirkkeitä
juonenkuljetukseen ja informaatioaskeleiden sitomiseen toisiinsa.
Tekstikappaleiden keskiosat, samoin kuin lukujen keskiosat selostavat
asiaa vähemmän tuntevalle yksityiskohtia, kun taas aihepiirissä jo
sisällä olevat lukijat voivat alkuvirkkeitä silmäilemällä edetä
tekstissä tehokkaasti eksymättä tarinan juonesta.

Kullakin kirjoittajalla on oma temponsa, joka välittyy lukijalle
tekstikappaleiden pituudessa ja niihin sisällytettyjen ajatuskulkujen
mutkikkuudessa. Kussakin tekstikappaleessa pitäisi pitäytyä vain yhdessä
informaatioaskelessa tai olennaisessa päättelyaskelessa, muuten juonen
seuraaminen käy raskaaksi olennaisten lauseiden etsiskelyksi. Yksivirkkeisiä
tekstikappaleita on syytä varoa. 



\section{Lähdeviittausten käyttö}


Olennaisia opittavia asioita viittaustekniikoissa ovat viitteen paikka
tekstissä, oikea lähdeluettelojärjestys valitun viitetyylin parina sekä
taito ja tahto noudattaa annettua tyylimääräystä. Väitöskirjoissa ja
lehti- tai konferenssiartikkeleissa tekstin hyväksyminen riippuu myös
näiden yksityiskohtien asianmukaisesta käsittelystä. Tästä syystä
laitoksella nähdään tarpeelliseksi opiskelijoiden tutustua edes
pinnallisesti myös muihin tyylilajeihin ja oppia käyttämään
automatisoituja muotoilutyökaluja tehokkaasti, jolloin tyylimuutokset
ovat tehokkaita.

Lähdeviitteet sijoitetaan aina virkkeen sisäpuolelle. Siten esimerkiksi
tekstikappaleen lopussa irrallaan oleva viite ei ole asiallinen. Tilanne
ei muutu, vaikka viite sujautettaisiin tekstikappaleen viimeisen
virkkeen sisään. 
Lähdeviittauksen yhteyteen merkitään mukaan tarkentavat
sivunumerot, mikäli lukijan olisi työlästä löytää asianomainen kohta
viitatusta lähteestä. 


Tehokkaita viitteensijoittelupaikkoja ovat esimerkiksi uuden käsitteen
nimeämiskohta ja virkkeen loppu kun kyseessä on lähteestä lainattu
väite. On myös muistettava lainausmerkkien käyttö silloin kun tehdään
suoria lainauksia.

Tekstin jäsentelyn on tuotava selkeästi esiin, mihin asiaan viite
liittyy. Samalla tulee ymmärrettäväksi se, kuinka pitkään
tekstikatkelmaan ko. viite liitetään. Ei ole siten asiallista aloittaa
lukua nimeämällä yhtä tai useampaa lähdettä luvun taustaksi, vaan
viitteitä on kiinnitettävä täsmällisemmin väitteisiin ja käsitteisiin.
Luvun avaus viitetiedolla voi olla oire myös suuremmasta ongelmasta:
lähderiippuvuudesta. Aloitteleva kirjoittaja helposti toistaa lähteestä
oppimaansa ilman että tarpeellinen analysointi ja prosessointi suhteessa
muuhun opittuun olisi vielä tapahtut.

Viitteillä ja sanamuodolilla on
myös tuotava selkeästi esiin se, mikä teksissä on lainattua ja mikä oman
pohdinnan ja valikoinnin tulosta.


Lähdeviittauksiin käytetään Tietojenkäsittelytieteen osastolla
numeroitua tyyliä ja APA-tyyliä, valinnan näiden välillä tekevät kunkin
ryhmän valvoja ja ohjaaja yhdessä. 
Numeroitu tyyli on esimerkiksi IEEE- ja
ACM-julkaisuissa yleisesti käytetty ja puolustaa siten paikkaansa.
APA-tyyli on poikkeuksellinen ns. kovissa tieteissä, mutta monet
valvojista pitävät siitä sen luettavuuden vuoksi. Numeroita joutuu nimiä
useammin tarkistamaan lähdeluettelosta, sillä tarkastus- ja
arvointiprosessiin kuuluu arvioida myös lähteiden valitaa ja niiden
käyttötapaa.





\section{Yhteenveto}

Yhteenveto  vaatimattomimmillaan on vain lyhyt kertaus kirjoituksen
keskeisistä asioista. Arvokkaamman yhteenvedon saa aikaan kommentoimalla
 työn tulosten arvoa, työn liittymistä ympäristöön ja
tulevaisuudennäkymiä. Tällaiset arviot  huolellisesti
perusteltava.

\section{Lähdeluettelon laatiminen}

Tieteellisen kirjoittamisen kurssin töiden lähdeluetteloiden
laatimisessa noudatetaan seuraavia ohjeita.

Niiden taustalla on kaksi
keskeistä pyrkimystä: tehdä viitatun lähteen hankkiminen luettavaksi
mahdollisimman helpoksi ja ilmaista, millaisen arviointiprosessin
läpi käyneeseen kirjoitukseen vedotaan.
Näistä syistä
\begin{itemize}
\item lähdeviitteen tulee aina olla niin tarkka, että
lähde on sen perusteella tunnistettavissa ja löydettävissä luetteloista
ja kirjastoista,
\item erityyppisten lähteiden (kirjat, konferenssit, lehdet) on erotuttava
toisistaan
ja
\item luettelon eri osien tulee olla mahdollisimman
yhdenmukaisia, erityisesti lähdetyypin sisällä.
\end{itemize}


Riippumatta käytettävästä viitetyylistä, 
lähteet ovat Tietojenkäsittelytieteen osaston opinnäytteiden lähdeluetteloissa tekijän nimen mukaisessa aakkosjärjestyksessä,
saman tekijän (tekijäryhmän) työt julkaisuajan mukaisessa
järjestyksessä. Jos jollakin lähteellä ei ole henkilötekijää, se
aakkostetaan julkaisun nimen mukaisesti. 

Kustakin lähteestä annetaan seuraavat tiedot, edelleen viitetyylistä riippumatta:
\begin{itemize}
\item (tarvittaessa lähdeviitelyhenne).
\item
tekijän tai tekijöiden nimet (sukunimi, etunimien alkukirjaimet)
alkuperäisessä järjestyksessään; jos tekijöitä on enemmän kuin kolme,
voidaan toimia siten, että
vain ensimmäinen tekijä nimetään ja muiden tilalle kirjoitetaan {\em et
al.}
\item
julkaisun tai artikkelin nimi alkuperäisessä muodossaan
\item
julkaisupaikan tiedot:
\begin{itemize}
\item
kirjasta: kustantaja, julkaisupaikka (voidaan jättää pois, jos kyseessä
on tunnettu kustantaja), vuosi ja
\item
lehtiartikkelista: lehden nimi, volyymi, numero, vuosiluku ja kuukausi (suluissa),

\item
artikkelikokoelmassa (esim. konferenssijulkaisussa) ilmestyneestä
artikkelista:
\begin{itemize}
\item kokoelman nimi, toimittaja, kustantaja, julkaisupaikka ja vuosi
{\em tai}
\item konferenssin nimi, järjestäjä, paikka ja aika,
\end{itemize}
\item
raportista: julkaisusarja, raportin numero, julkaisupaikka, julkaisija ja vuosi
ja
\item
www-lähteestä: verkko-osoite, voimassaoloajankohta, mahdollisesti
viittausajankohta hakasuluissa
\end{itemize}
\item
sivunumerot, mikäli lähteenä käytetty julkaisu on artikkeli tai kokoomateoksen itsenäinen luku.
\end{itemize}

Normaaliin suomalaiseen tapaan artikkelin nimessä ainoastaan
ensimmäinen sana kirjoitetaan isolla alkukirjaimella, sen sijaan
konferenssien ja kokoelmajulkaisujen nimissä käytetään isoa
alkukirjainta jokaisen sanan alussa (artikkelisanoja ja prepositioita
lukuunottamatta). Katso mallia oheisista esimerkeistä.
Kokoelman nimen edessä on syytä selvyyden vuoksi käyttää sanaa {\em
Teoksessa}, paitsi kun on kysymys konferenssijulkaisusta, jonka nimi
alkaa lyhenteellä {\em Proc.} (sanasta Proceedings). Tällöin ei tarvita
mitään täydennystä.
Tämän eron näkee esimerkiksi vertaamalla
lähdeviitteiden~''\citep{dantowsley90}''
ja~''\citep{gannonetal89}'' ulkoasuja.

WWW-lähteiden käytössä on syytä muistaa, että verkossa julkaisukynnys on
olematon. Kannattaa siten keskittyä tunnettujen tieteellisten
kustantajien julkaisuihin ja niihin teknisiin standardeihin, joille WWW
on ainoa julkaisukanava. Mikäli sama julkaisu on saatavissa myös
perinteisessä muodossa, viitataan ensisijaisesti siihen ja käytetään
verkko-osoitetta lisätietona. Lähdeluettelossa on annettu esimerkit
useita kanavia julkaistusta kirjoituksesta~\citep{abiteboul,dietinger} sekä pelkästään
WWW-julkaisuna
leviävästä standardista~\citep{bray}.

Erityisesti varoitetaan Wikipedian käytöstä tieteellisessä tekstissä.
Vaikka sen avulla on helppo alustavasti tutustua joihin aihepiireihin ja
asiantuteva lukija voisi teksin kelvolliseksi tiettynä hetkenä
hyväksyäkin, ei se foorumina millään lailla täytä tieteellisesti
vertaisarvoidun tutkimusfoorumin kriteerejä.
Jos Wikipedia-artikkelia  ei mitenkään malta ajankuvana olla mainitsematta, käytettäköön jotain muuta kuin lähdeviitetekniikkaa tähän taiteelliseen otteeseen, vaikkapa alaviitteitä. Olennaista silloinkin on, että tieteellinen sisältö ei tule tällä korvatuksi vaan sen puute korostetuksi.

WWW-lähteeseen viittaamisessa pätevät samat periaatteet kuin
perinteisiin lähteisiin viitattaessa: lähdeviitteessä ilmaistaan
otsakkeet, kirjoittajat, toimittajat js muut seikat. Eroa on ainoastaan
verkko-osoitteen ja sen voimassaoloajankohdan ilmaisemisessa. Mikäli
lähde on julkaistu ainoastaan verkossa, voidaan web-osoitetta (URL)
käyttää vastaavasti kuin perinteisen julkaisun paikannusinformaatiota
(lehden ja se numeron julkaisutiedot). Lähdeluettelossa on WWW-viittausten yhteydessä aina syytä ilmaista päivämäärä, jolloin linkin voimassaolo ja lähteen sisältö on tarkastettu.
Esimerkkeinä verkkoviitteistä soveltuvat seuraavat:
\begin{itemize}
\item Gergen, Kenneth (1999) Narrative, Moral Identity and Historical
Consciousness: a Social Constructionist Account.
http://www.swarthmore.edu/SocSci/kgergen1/text3.html. Haettu 11.6.1999.
\item 	
Ritala-Koskinen, Aino and Valokivi, Heli (2006) The Role of Development
Skills in Social Work Practice Education in Finland. Social Work and
Society, The International Online-Only Journal 4(2006)1.
http://www.socwork.net/2006/1/series/transition/ritalakoskinenvalokivi.
Viitattu 30.8.2006.
\item Heinisuo, Rami and Ekholm, Kai (1997) Elektronisen viittaamisen
opas. Jyväskylän yliopiston kirjaston julkaisuja n:o 40. Jyväskylä:
Jyväskylän yliopiston kirjasto. http://www.pori.tut.fi/~multisil/evo/.
Viitattu 29.8.2006.
\end{itemize}

Kirjoituksen lähdeluettelossa luetellaan täsmälleen ne lähteet, joihin
viitataan kirjoituksen tekstiosassa. Tämän kirjoituksen lähdeluettelo on
tarkoitettu lähinnä esitystavan esimerkiksi, mistä syystä siinä on
''ylimääräisiä'' lähteitä.


Pääsääntöisesti julkaisun tai artikkelin nimen perään tulee piste,
samoin kunkin lähteen bibliografisten tietojen perään. Muut tiedot
erotetaan toisistaan pilkulla. Useimmissa tapauksissa  
voidaan noudattaa teknisten välineiden antamaa mallia, sillä edellytyksellä, että ylläolevat vaatimukset muuten täyttyvät.



\chapter{Ulkoasulliset seikat}

  Tässä luvussa käsitellään yleisimpiä
tekstin tekniseen esittämiseen liittyviä seikkoja.  

\section{Työn osien järjestys}

   Kirjoituksen alussa on aina
erillinen, mallin mukainen kansilehti. Toisena sivuna on
tiivistelmäsivu, sen jälkeen sisällysluettelo (yksi tai useampia
sivuja) ja sitten varsinainen teksti. Sivunumerointi aloitetaan vasta
ensimmäiseltä tekstisivulta (arabialaisella ykkösellä). (Tarkat
jättävät ykkössivun numeromerkittä.) Sisällysluetteloon merkitään
kaikki (numeroidut) otsikot ja vastaavat sivunumerot. Monet
tekstin\-käsittelyjärjestelmät muodostavat itse sisällysluettelon,
jolloin kirjoittajan ei tarvitse huolehtia luettelon sivunumeroiden
päivittämisestä tekstin kehittyessä. Sisällysluettelo\-sivu ja sitä
edeltävät sivut voidaan haluttaessa numeroida erikseen (roomalaisin
numeroin) esimerkiksi tämän mallin mukaisesti.  

Varsinaisen tekstin
jäljessä, mutta itse työhön kuuluvana on ensimmäisenä lähdeluettelo,
jonka otsikkoa ei numeroida. Lähdeluettelon jälkeen sijoitetaan
mahdolliset liitteet, jotka otsikoidaan ja varustetaan sisäisillä
sivunumeroilla.  



Mikäli kuvista, algoritmeista
ja taulukoista halutaan tehdä yhtenäinen luettelo, sijoitetaan
luettelot sisällysluettelon jälkeen. Luetteloiden käyttöarvosta on
eriäviä mielipiteitä, joten niiden laatimiseen ei varsinkaan ilman
tekstinkäsittelyjärjestelmän tukea kannata ryhtyä ilman tarkastajan
erityistä toivetta.  

Mikäli kirjoitukseen erityissyistä halutaan
liittää aakkosellinen hakemisto, sijoitetaan se lähdeluettelon jälkeen
ennen liitteitä. Indeksi merkitään sisällysluetteloon samoin kuin
lähdeluettelo (numeroimaton luku). Mikäli indeksin tekemiseen
ryhdytään, on syytä käyttää tekstinkäsittelyjärjestelmän tarjoamaa
automatiikkaa.

Teksin luonnollisen juonenkuljetuksen mukana esiin
tulevien käsitteiden määrittelyjen sijasta ei pidä yrittää sen enempää
pakata kaikkia määritelmiä johdantoon kuin laatia johdantoa ennen
käsitelistaa tai lyhenteiden selityslistaa. Kumpikaan ei sovi
tavanomaiseen argumentoivaan tieteelliseen tekstityyliin, vaikka
teknisessä yhteydessä niillä liitteinä voi olla lisäarvoa.


\section{Tekstin yleinen sijoittelu}

Lopullinen tutkielmaversio voi olla yksi- tai kaksipuoleiseksi aseteltua
ja riviväliltään 1,5 tai 1.  Erityyppisissä
teksteissä haasteet ja asetteluvaatimukset voivat olla erilaiset. Erota
kappaleet toisistaan yhdellä tyhjällä rivillä tai
käytä tekstinkäsittelytyökalujen ominaisuuksia
hyödyksesi ja määrittele tekstikappaleiden väliin jäävä tila hieman
normaalia riviväliä suuremmaksi.

Kirjoituksen lukujen, kuvien ja taulukoiden erottumisen kannalta
tärkein keino on riittävän tilan käyttö niiden ympärillä. Kuvan ja
nimekkeen tulee olla selkeästi yksi kokonaisuus, joka eroaa muusta
tyhjän tilan rajaamana. Kuvan tai taulukon on aina numerointinsa ja
nimekkeensä kanssa mahduttava yhdelle sivulle tai varmasti
kaksipuolisena paperidokumenttina tarkasteltavassa tekstissä aukeamalle.
Kuvissa fonttikoko ei saa alittaa 8 pistettä.



Jos uusi luku tulisi alkamaan aivan sivun alareunasta (vain yksi tai
kaksi riviä varsinaista tekstiä), aloita mieluummin uusi sivu. Jokaista
uutta lukua ei kuitenkaan ole tarpeen --- etenkään lyhyessä
kirjoituksessa --- aloittaa uudelta sivulta: jos kirjoituksessa on
paljon melkein tyhjiä sivuja, lukija voi epäillä, että kirjoittaja on
yrittänyt saada kirjoituksensa näyttämään pitemmältä kuin se onkaan. 
 Tyhjää tilaa kannattaa käyttää hyödyksi myös kuvien ja taulukoiden
yhteydessä. Erityisesti jos kirjoituksessa käytetään kauttaaltaan samaa
tekstityyppiä, tyhjät rivit ovat välttämättömiä erottamaan esimerkiksi
tekstiä ja taulukkoa toisistaan. Tyhjä tila on halpaa, mutta se lisää
selkeyttä ja luettavuutta.  


\section{Kuvat ja taulukot}


Kuva tai taulukko sijoitetaan mahdollisimman lähelle
(ensimmäistä) tekstikohtaa, jossa siihen viitataan, ei kuitenkaan
kyseistä viittausta aikaisemmaksi.
Tekstissä on syytä myös kertoa, mitä kuvalla halutaan havainnollistaa.
Kuvan voi lukea monella eri tavalla, joten lukijaa on ohjattava.

Kuvaa ei pidä sijoittaa välittömästi luvun otsikon alle, vaan on
aloitettava tekstillä. Kuvaa ei pidä sijoittaa keskelle tekstikappaletta
(saati virkettä), paitsi jos kuva tulee sivun alkuun tai loppuun eikä
kappaleen jatkumisesta tule epäselvyyttä.

Kuvan ei aina tarvitse olla välittömästi viittaavan kappaleen
perässä. Esimerkiksi viittauskohdan ja
vasta seuraavalle sivulle mahtuvan kuvan väliin jäävää sivun loppuosaa
ei jätetä tyhjäksi. Kuvaa ei kuitenkaan pidä viedä seuraavaa
sivua kauemmas viittauskohdasta.


Varsinaista kuvan esittämistä havainnollistaa kuva~\ref{kuvaesimerkki}.
Huomiota on kiinnitettävä kuvan osien ja tekstimerkintöjen näkyvyyteen,
kuvan numerointiin ja otsikointiin. 

\begin{figure}[ht]
%\begin{figure}[tbh] t= top, b = bottom, h=here
\ \newline
\begin{center}
\includegraphics[width=0.75\textwidth]{kuvaesimerkki.pdf}
\caption{Kuvan elementit.}
\label{kuvaesimerkki}
\end{center}
\end{figure}


Kuvien kokoon on kiinnitettävä huomiota. Käytettyjen merkintöjen
on oltava helposti luettavissa ja selkeät. Esimerkiksi
suorituskykykäyriä esitettäessä akselit on nimettävä, asteikot
merkittävä ja käytetyt yksiköt tuotava selkeästi esiin.
Samankaltaisia asioita esitettäessä useammalla kuvalla on
syytä käyttää samaa mittakaavaa vertailun helpottamiseksi.

Kuvan otsikko kirjoitetaan kuvan alle ja sen tulee olla mieluummin lyhyt
ja ytimekäs kuin liian selittelevä.
Samoin toimitaan taulukoiden otsikoinnissa.

Kuvat ja taulukot numeroidaan juoksevasti. Pitkissä teksteissä käytetään
kaksitasoista numerointia (esimerkiksi Kuva 3.1) pääluvuittain, lyhyissä
riittää yksitasoinen numerointi.


Kuva- ja taulukko-otsikoiden yhdenmukaiseen esitystyyliin on syytä kiinnittää
huomiota, samoin mm. välimerkkeihin. Luontevaa on käyttää
kuvatekstin lopussa pistettä, ovathan useimmat kuvateksteistä virkkeitä. 

(Kuvien ja taulukoiden otsikointityyli vaihtelee
kustantajittain ja julkaisuittain. Samoin tuntuu suositeltava käytäntö
Tietojenkäsittelytieteen laitoksen sisällä vaihtelevan taulukon
otsikon sijainnin suhteen.)


\section{Otsikot}

Otsikoissa voi käyttää muusta tekstistä poikkeavaa kirjasintyyppiä,
alleviivausta, suurempaa kirjasinkokoa tms.\ erotuskeinoa, yleensä
kuitenkin vain yhtä näistä, koska kovin monta erilaista kirjasintyyppiä
ja -kokoa tekee ulkoasusta helposti sekavan.  Otsikoiden esitystavan on
oltava johdonmukainen läpi koko kirjoituksen. Numeroimattomia
''ylimääräisiä'' otsikoita ei tule yleensä käyttää.


\section{Mallin käyttö}

Voit käyttää tätä kirjoitusta mallina oman opinnäytteesi ulkoasua
varten. Eri tekstinkäsittelyjärjestelmissä käytössä olevat yksityiskohdat kuten
kirjasintyypit ja -koot ja rivivälit  poikkeavat toisistaan, joten
pienet poikkeamat ovat toki hyväksyttäviä.

Tieteellisen kirjoittamisen kurssin luennoilla ja
liitteenä olevassa ohjeessa annetut töiden ohjeelliset sivumäärät
koskevat työtä, joka vastaa ulkoasultaan tätä ohjetta (kirjasinkoko
12~pistettä). Tässä tekstissä keskimääräinen rivin pituus lienee noin
80~merkkiä ja sivun pituus 35-40~riviä.
Sivumääriin lasketaan varsinaisen tekstiosuuden pituus ja lähdeluettelo
(arabialaisin numeroin numeroitu osuus), ei kansilehteä, tiivistelmää
eikä sisällysluetteloa. Sivumääräarviossa otetaan huomioon hyvin vajaat
sivut, joita syntyy paljon lyhyiden lukujen ja taittotyyliin määritellyn
luvun avauksen pakottaminen oikeanpuolimmaiselle sivulle. 

\chapter{Yhteenveto}

Tämän kirjoituksen tarkoituksena on toimia muistilistana eräistä
esitystavallisista säännöistä, joihin harjoitusten ja tutkielmien
kohdalla on syytä kiinnittää huomiota.

Annetut ohjeet on laitoksen henkilökunta muotoillut yhdessä keskustellen
ja noudattaen oman tieteenalansa perinteitä. Eri erikoistumisaloilla ja
erilaisillaa määräävässä asemassa olevissa julkaisufooruilla käytänteet
vaihtelevat ja nuorten tutkijoiden onkin tiedostettava ero yleisten
sisältöohjeiden ja teknisten muotoilusääntöjen välillä. Aina tekstin
valmistuessa on tarkastettava erikseen, täyttääkö se annetut
pituusrajoitteet ja vastaako se annettuja muotoiluohjeita, olivatpa ne
kuinka pikkutarkkoja tahansa. Tarkasta sääntöjen noudattamisesta syntyy
yhteinäisyyttä kokoovan julkaisun tasolla, mikä helpottaa lukijoiden
työskentelyä.

Tämä ohje vastaa vain asettelullisiin kysymyksiin ja sen rinnalla on
syytä tutustua materiaaliin ja luentoihin, joissa keskitytään tekstin
varsinaiseen sisältöön. Olennaisin väline on kuitenkin akateemisesti
pidemmälle ehtineen, jo julkaisuja rakentaneen ohjaajan palaute ja
mentorointi.

