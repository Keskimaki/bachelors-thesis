\chapter{Yhteenveto\label{summary}}

DevOps-toimintamalli perustuu jatkuvaan integraatioon ja toimitukseen sekä toimintamallin vaiheiden mukaiseen ohjelmistotuotantoon.
Jatkuvan integraation ja toimituksen osana koodimuutokset testataan ja viedään testi- ja tuotantoympäristöön nopealla julkaisusyklillä.
Näin uudet koodimuutokset saadaan nopeasti käyttöön ja mahdolliset ongelmatilanteet huomataan nopeasti.

Konttiteknologia ja konttien orkestrointi tukevat DevOps-toimintamallin mukaista ohjelmistotuotantoa.
Konttiteknologian avulla samankaltaista konttia voidaan käyttää ensin testauksessa osana jatkuvaa integraatiota ja myöhemmin tuotantokäytössä konttiorkestrointialustalla.

% Rewrite
Konttien orkestrointi mahdollistaa muun muassa resurssien käytön hallinnan, konttien monitoroinnin ja virhetilanteista toipumisen.
Näin konttiorkestrointialustat tukevat DevOps-toimintamallin eri vaiheita.

Virtuaalikoneet mahdollistavat konttiteknologian tapaan vakaan ja toistettavan ympäristön.
Kontteja hitaamman käynnistysnopeuden ja suuremman kokonsa vuoksi ne eivät kuitenkaan sovellu yhtä hyvin orkestrointialustojen käyttöön.
Virtuaalikoneet tarjoavat kuitenkin kontteja paremman eristyneisyyden ja tietoturvan.
Pilvialustat tarjoavat monia DevOps-toimintamallia tukevia palveluita, jotka täydentävät tai osaltaan vähentävät tarvetta konttien orkestroinnille.

Norppa-palautejärjestelmä tarjoaa käytännön esimerkin DevOps-toimintamallin mukaisesta ohjelmistotuotannosta ja konttien orkestroinnin käytöstä.
DevOps-toimintamallin mukainen ohjelmistotuotanto mahdollistaa esitetyn nopean julkaisuputken ja laadunvalvonnan.
Konttiteknologian ja konttien orkestroinnin mahdollistama ohjelmistoarkkitehtuuri tukee saman palvelun käyttöä ja jatkokehitykstä Helsingin ja Tampereen yliopistojen välillä.
