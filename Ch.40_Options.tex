\chapter{Muut vaihtoehdot\label{options}}

Konttiteknologian ja konttien orkestraation lisäksi käytössä on muita julkaisutapoja. Konttiteknologian sijaan suoritusympäristön virtualisaatio voidaan toteuttaa virtuaalikoneiden avulla tai palvelua voidaan suorittaa suoraan fyysisellä palvelimella \cite{Watada19}.
Sekä konttien, että virtuaalikoneiden hallinnointiin voidaan käyttää konttiorkestraatioalustan sijaan erilaisia pilvialustoja \cite{Bousselmi14}.
Konttiteknologioiden käytön yleistyminen on myös johtanut uusiin niin sanottuihin Serverless-ratkaisuihin \cite{Baldini17}.

\section{Virtuaalikoneet}

Virtualisaatiolla tarkoitetaan fyysisen laitteen tai resurssin toteuttamista virtuaalisessa muodossa.
Kokonaista virtualisoitua käyttöjärjestelmää kutsutaan virtuaalikoneeksi.
Virtuaalikoneiden avulla yksittäisen fyysisen palvelimen resurssit voidaan jakaa useamman virtuaalisen käyttöjärjestelmän välillä \cite{Smith05}.
Virtuaalikoneiden suoritus tapahtuu virtualisointiympäristössä (engl. hypervisor, virtual machine monitor), joka huolehtii fyysisen palvelimen resurssien abstrahoinnista ja virtuaalikoneiden hallinnoinnista \cite{desai13}.

Virtuaalikoneet mahdollistavat konttien tapaan palvelimesta erillisen toistettavan ympäristön.
Kuvassa \ref{fig:container} esitetään virtuaalikonejulkaisun rakenne.
Luvussa \ref{container} todettiin, että kontti on kokonsa ja käynnistysnopeutensa suhteen virtuaalikoneita tehokkaampi ratkaisu.
Virtuaalikoneiden rakenteelliset erot kontteihin nähden mahdollistavat kuitenkin paremman tietoturvan \cite{Sultan19}.

% What are VMs?
% Benefits, downsides, security
% No orchestration

\section{Fyysiset palvelimet}

% bare metal
% downsides, simplicity and security benefits
% performance?

\section{Pilvialustat}

% Cloud services etc.
% Can orchestrate VMs, containers, bare metal

\section{Serverless}
% Own Section?
% Serverless, new thing, acually containerized?
