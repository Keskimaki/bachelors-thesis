\chapter{Muut julkaisutavat ja orkestrointiratkaisut\label{options}}

Konttiteknologian lisäksi käytössä on muita julkaisutapoja.
Konttiteknologian sijaan suoritusympäristön virtualisaatio voidaan toteuttaa virtuaalikoneiden avulla tai palvelua voidaan suorittaa suoraan fyysisellä palvelimella \cite{Watada19}.
Sekä konttien että virtuaalikoneiden hallinnointiin voidaan käyttää konttiorkestrointialustan sijaan tai lisäksi erilaisia pilvialustoja \cite{Bousselmi14}.
Konttiteknologian käytön yleistyminen on myös johtanut uusiin palvelimettomaan arkkitehtuurin perustuviin ratkaisuihin \cite{Baldini17}.

Käsiteltävät vaihtoehdot jaetaan tässä tutkielmassa julkaisutapoihin, joita ovat kuvassa \ref{fig:container} esitetyt fyysiselle palvelimelle julkaisu sekä virtuaalikone- ja konttipohjaiset julkaisut.
Julkaisutapojen lisäksi käsitellään erilaisia orkestrointiratkaisuja konttien orkestroinnin ja pilvialustojen tarjoamien palveluiden sekä palvelimettoman arkkitehtuurin muodossa.

\section{Virtuaalikoneet}

Virtualisoinnilla tarkoitetaan fyysisen laitteen tai resurssin toteuttamista virtuaalisessa muodossa.
Kokonaista virtualisoitua käyttöjärjestelmää kutsutaan virtuaalikoneeksi.
Virtuaalikoneiden avulla yksittäisen fyysisen palvelimen resurssit voidaan jakaa useamman virtuaalisen käyttöjärjestelmän välillä \cite{Smith05}.
Virtuaalikoneiden suoritus tapahtuu virtualisointiympäristössä (\textit{hypervisor}, \textit{virtual machine monitor}), joka huolehtii fyysisen palvelimen resurssien abstrahoinnista ja virtuaalikoneiden hallinnoinnista \cite{desai13}.

Konttiteknologia on käyttöjärjestelmätason virtualisointia, kun taas virtuaalikoneet ovat laitteistotason virtualisointia \cite{Compastie20}.
Virtuaalikoneet mahdollistavat konttien tapaan palvelimesta erillisen toistettavan ympäristön.
Kuvassa \ref{fig:container} esitetään palvelun omat ja jaetut resurssit eri julkaisutavoilla.
Virtuaalikonejulkaisun osalta jokaisella virtualisointiympäristössä toimivalla virtuaalikoneella on oltava oma kopionsa käyttöjärjestelmän ytimestä.
Konttijulkaisussa konttien suoritusympäristössä toimivat kontit taas jakavat saman käyttöjärjestelmän ytimen.

\begin{figure}[ht]
\begin{center}
\includegraphics[width=1\textwidth]{figures/container_evolution.png}
\caption{Sovelluksen omat ja jaetut resurssit eri julkaisutavoilla \cite{Kubernetes23}\label{fig:container}.}
\end{center}
\end{figure}

\pagebreak

Tämän rakenteellisen eron vuoksi kontti on kokonsa ja käynnistysnopeutensa suhteen virtuaalikonetta tehokkaampi ratkaisu.
Virtuaalikoneiden erilliset käyttöjärjestelmät luovat kuitenkin paremman eristyneisyyden (\textit{isolation}), jonka vuoksi virtuaalikoneet tarjoavat kontteja paremman tietoturvan \cite{Sultan19}.

Virtuaalikoneet voivat olla kontteja sopivampi julkaisuratkaisu varsinkin tietoturvan kannalta kriittisissä tilanteissa.
Lisäksi virtuaalikoneet tukevat useita käyttöjärjestelmiä, kuten Windows- ja macOS-pohjaisia virtuaalikoneita, kun taas konttiteknologia on kehitetty pääasiassa Linux-pohjaisille järjestelmille \cite{Watada19}.
Virtuaalikoneiden kontteja suuremmat resurssikustannukset ja hitaampi käynnistymisaika kuitenkin tekevät niistä huonommin soveltuvia konttiorkestrointialustojen kaltaisiin ratkaisuihin.
Useat pilvialustat tarjoavat virtuaalikoneille konttien orkestrointia vastaavia palveluita, mutta Kuberneteksen kaltaista vakintuunutta ja yhtä kattavaa järjestelmää ei ole kehitetty. % Can't find citation

DevOps-toimintamallin vaiheiden kannalta tietoturva on tärkeintä operointi- ja monitorointivaiheissa, joiden osalta palvelun suoritusympäristönä voi toimia virtuaalikone.
Konttiteknologiaa voidaan kuitenkin käyttää muissa vaiheissa toimintamallia ja mahdollisesti myös operointi- ja monitorointivaiheissa virtuaalikoneen sisällä.
Tässä tilanteessa menetetään kuitenkin luvussa \ref{platforms} mainitut konttien orkestroinnin tarjoamat edut.

\section{Fyysiset palvelimet}

Virtualisointi, sekä virtuaalikoneiden, että konttien muodossa on hyödyllistä monissa ohjelmistotuotannon tilanteissa.
Julkaisuratkaisuna voidaan kuitenkin käyttää myös palvelun suorittamista suoraan fyysisellä laitteistolla.
Kuva \ref{fig:container} esittää perinteisen julkaisutavan resurssien jaon.
Kontti- ja virtuaalikonejulkaisuista poiketen kaikki sovelluksen ulkopuoliset resurssit ovat jaettuja ja erillistä virtualisointiympäristöä tai konttien suoritusympäristöä ei tarvita. 
Tämän kaltainen perinteinen julkaisutapa on suorituskyvyltään ja tilavaatimukseltaan tehokkain ratkaisu yksittäisen palvelun julkaisussa, jonka vuoksi sitä käytetään varsinkin sulautetuissa järjestelmissä \cite{Heiser08}.

Virtualisaatio varsinkin konttiteknologian osalta ei kuitenkaan aiheuta merkittäviä kustannuksia \cite{torrez19}.
Perinteinen julkaisutapa ei myöskään tarjoa virtuaalikonejulkaisun kaltaista eristyneisyyttä tai luvussa \ref{orchestration:devops} käsiteltyjä konttiteknologian etuja.
Julkaisutapa ei näin ollen sovellu hyvin DevOps-toimintamallin mukaiseen ohjelmistokehitykseen.

\section{Pilvialustat ja palvelimeton arkkitehtuuri}

Pilvialustojen käytöllä tarkoitetaan palvelun tai sen osan toteuttamiseen ulkoisen palveluntarjoajan järjestelmien avulla.
Pilvialustan käytöllä monia ohjelmistotuotannon ja DevOps-toimintallin vaiheita voidaan tarvittaessa ulkoistaa pilvialustan vastuulle \cite{tomarchio20}.
Erilaisia pilvialustoja on paljon ja niiden toimintatavat ja tarjoamat palvelut vaihtelevat laajalti.
Tämän vuoksi tässä tutkielmassa pilvialustoja käsitellään käytännön esimerkkinä toimivan \textit{Google Cloud Platform}:in (GCP) kautta, joka on laajalti käytetty ja tarjoaa kattavan määrän erilaisia pilvialustoille ominaisia palveluita \cite{ahuja20}.

GCP tarjoaa muun muassa fyysisiä palvelimia, tallennustilaa, virtuaalikoneita, tietokantoja ja Kubernetes-klustereita \cite{Products23}.
Näin ollen kaikki aiemmin käsitellyt julkaisutavat sekä konttien orkestrointiin perustuvat ratkaisut voidaan toteuttaa pilvialustan avulla.
Pilvialustan käyttö ei siis ole myös konttiorkestrointialustan käyttöä poissulkeva ratkaisu.
Luvussa \ref{platforms} mainittu Kuberneteksen \textit{Cloud controller manager} mahdollistaa pilvialustojen ja konttien orkestroinnin etujen yhdistämisen.

Kuva \ref{fig:architecture} esittää esimerkkiarkitehtuurin Kubernetes-klusterin ja GCP-pilvialusta tarjoamien palveluiden yhdistämisestä.
Arkkitehtuuri sisältää luvussa \ref{orchestration} käsitellyt Kubernetes-klusterin osat.
Pilvialusta taas tarjoaa muun muassa tietoliikenneratkaisun, pysyväistallennustilaa ja klusterin ulkopuolisen kuormituksen tasaajan.

% Cloud services etc.
% Can orchestrate VMs, containers, bare metal
% Serverless, new thing, acually containerized?

\begin{figure}[ht]
\begin{center}
\includegraphics[width=1\textwidth]{figures/gke_architecture.png}
\caption{Kubernetes-klusteri yhdistettynä GCP:n tarjoamiin palveluihin \cite{cluster23}\label{fig:architecture}.}
\end{center}
\end{figure}

GCP tarjoaa myös palvelimettomaan arkkitehtuuriin (\textit{Serverless}) perustuvana ratkaisuna \textit{Cloud Run} nimisen konttien suoritusympäristön, jonka tavoitteena on poistaa kaikki tarve ohjelmistoinfrastruktuurin hallinnoinnille \cite{Products23}.
Palvelimeton arkkitehtuuri tarkoittaa pilvialustan infrastruktuurin piilottamista käyttäjältä ja palveluiden automaattista skaalausta käyttömäärän mukaisesti.
Tämä mahdollistaa laskutuksen käyttömäärän perusteella, kun taas muut pilvialustapalvelut aiheuttavat usein jatkuvia kustannuksia \cite{shafiei22}.
Palvelimettomaan arkkitehtuuriin perustuvat ratkaisut pohjautuvat yleensä konttiteknologian käyttöön ja niiden haittapuolia ovat ennalta-arvaamaton resurssien käyttö ja hidas käynnistymisaika \cite{shafiei22, mondal22}.

Palvelimeton arkkitehtuuri soveltuu varsinkin tilanteisiin, joissa julkaistavan palvelun käyttömäärät ovat epätasaisia, koska jatkuvasti käynnissä oleva palvelin ei aiheuta turhia kustannuksia.
Palvelimeton arkkitehtuuri voi olla myös soveltuva ratkaisu yksittäisten kontitettujen palveluiden julkaisuun, kun luvussa \ref{platforms} käsiteltyjä konttien orkestrointipalveluita ei tarvita.
Myös palvelimeton arkkitehtuuri on tarvittaessa mahdollista yhdistää Kubernetes-pohjaisen konttien orkestroinnin kanssa \cite{mondal22}.

DevOps-toimintamallin kannalta pilvialustat yksinkertaistavat tarvittavien palveluiden, kuten jatkuvan integraation ja toimituksen käyttöä \cite{Singh19}.
Pilvialustoihin voi kuitenkin liittyä esimerkiksi lainsäädännöllisiä ongelmia muun muassa käyttäjätietojen säilömisen osalta \cite{Barati22}.
Konttiorkestrointialusta, esimerkiksi Kuberneteksen muodossa, on taas avoimen lähdekoodin standardisoitu ratkaisu, joka luvun \ref{orchestration:devops} perusteella soveltuu hyvin DevOps-toimintamallin mukaiseen ohjelmistotuotantoon.
Kubernetes-klusteri voidaan julkaista jopa yksittäiselle omalle palvelimelle tai muun muassa pilvialustan avulla \cite{Muddinagiri19, Khan22}.
