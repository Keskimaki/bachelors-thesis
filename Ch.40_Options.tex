\chapter{Muut vaihtoehdot\label{options}}

Konttiteknologian ja konttien orkestraation lisäksi käytössä on muita julkaisutapoja. Konttiteknologian sijaan suoritusympäristön virtualisaatio voidaan toteuttaa virtuaalikoneiden avulla tai palvelua voidaan suorittaa suoraan fyysisellä palvelimella \cite{Watada19}.
Sekä konttien, että virtuaalikoneiden hallinnointiin voidaan käyttää konttiorkestraatioalustan sijaan tai lisäksi erilaisia pilvialustoja \cite{Bousselmi14}.
Konttiteknologioiden käytön yleistyminen on myös johtanut uusiin niin sanottuihin Serverless-ratkaisuihin \cite{Baldini17}.

Käsiteltävät vaihtoehdot jaetaan tässä tutkielmassa julkaisutapoihin, joita ovat kuvassa \ref{fig:container} esitetyt fyysiselle palvelimelle julkaisu sekä virtuaalikone- ja konttipohjaiset julkaisut.
Julkaisutapojen lisäksi käsitellään erilaisia orkestrointiratkaisuja konttien orkestroinnin ja pilvialustojen tarjoamien palveluiden muodossa.
Vaihtoehtoja ja niiden soveltuvuutta DevOps-toimintamallin mukaiseen ohjelmistotuotantoon verrataan toisiinsa.

\section{Virtuaalikoneet}

Virtualisoinnilla tarkoitetaan fyysisen laitteen tai resurssin toteuttamista virtuaalisessa muodossa.
Kokonaista virtualisoitua käyttöjärjestelmää kutsutaan virtuaalikoneeksi.
Virtuaalikoneiden avulla yksittäisen fyysisen palvelimen resurssit voidaan jakaa useamman virtuaalisen käyttöjärjestelmän välillä \cite{Smith05}.
Virtuaalikoneiden suoritus tapahtuu virtualisointiympäristössä (engl. \textit{hypervisor}, \textit{virtual machine monitor}), joka huolehtii fyysisen palvelimen resurssien abstrahoinnista ja virtuaalikoneiden hallinnoinnista \cite{desai13}.

Virtuaalikoneet mahdollistavat konttien tapaan palvelimesta erillisen toistettavan ympäristön.
Kuvassa \ref{fig:container} esitetään virtuaalikonejulkaisun rakenne.
Luvussa \ref{container} todettiin, että kontti on kokonsa ja käynnistysnopeutensa suhteen virtuaalikoneita tehokkaampi ratkaisu.
Virtuaalikoneiden rakenteelliset erot kontteihin nähden mahdollistavat kuitenkin paremman tietoturvan \cite{Sultan19}.

% What are VMs?
% Benefits, downsides, security
% No orchestration

\section{Fyysiset palvelimet}

Virtualisointi, sekä virtuaalikoneiden, että konttien muodossa tarjoaa etuja monissa ohjelmistotuotannon tilanteissa.
Julkaisuratkaisuna voidaan kuitenkin käyttää myös palvelun suorittamista suoraan fyysisellä laitteistolla.
Tämän kaltainen perinteinen julkaisutapa on suorituskyvyltään tehokkain ratkaisu, jonka vuoksi sitä käytetään varsinkin sulatetuissa järjestelmissä \cite{Heiser08}.
Julkaisutapa ei sovellu hyvin DevOps-toimintamallin mukaiseen ohjelmistokehitykseen muun muassa luvussa \ref{orchestration:devops} käsiteltyjen etujen menettämisen vuoksi, joten julkaisutapaa ei käsitellä tässä tutkielmassa enempää.

% bare metal
% downsides, simplicity and security benefits
% performance?

\section{Pilvialustat ja Serverless}

Pilvialustojen käytöllä viitataan palvelun tai sen osan toteuttamiseen ulkoisen palveluntarjoajan järjestelmien avulla.
Pilvialustan käytöllä monia ohjelmistontuotannon ja DevOps-toimintallin vaiheita voidaan tarvittaessa ulkoistaa pilvialustan vastuulle \cite{tomarchio20}.
Erilaisia pilvialustoja on paljon ja niiden toimintatavat ja tarjoamat palvelut vaihtelevat laajalti.
Tämän vuoksi tässä tutkielmassa pilvialustoja käsitellään käytännön esimerkkinä toimivan \textit{Google Cloud Platform}:in (GCP) kautta, joka on laajalti käytetty ja tarjoaa kattavan määrän erilaisia pilvialustoille ominaisia palveluita \cite{ahuja20}.

GCP tarjoaa muun muassa fyysisiä palvelimia, tallennustilaa, virtuaalikoneita, tietokantoja ja Kubernetes-klustereita \cite{Products23}.
Näin ollen kaikki aiemmin käsitellyt julkaisutavat sekä konttien orkestrointiin perustuvat ratkaisut voidaan toteuttaa pilvialustan avulla.
Pilvialustan käyttö ei siis ole sen lisäksi myös konttiorkestraatioalustan käyttöä poissulkeva ratkaisu.

GCP tarjoaa myös Serverless-ratkaisuna \textit{Cloud Run} nimisen konttien suoritusympäristön, joka tavoitteena on poistaa kaikki tarve ohjelmistoinfrastruktuurin hallinnoinnille \cite{Products23}.
Serverless tarkoittaa pilvialustan infrastruktuurin piilottamista käyttäjältä ja palveluiden automaattista skaalaamista käyttömäärän mukaisesti.
Tämä mahdollistaa laskutuksen käyttömäärän perusteella, kun taas muut pilvialustapalvelut aiheuttavat usein jatkuvia kustannuksia \cite{shafiei22}.

% Cloud services etc.
% Can orchestrate VMs, containers, bare metal
% Serverless, new thing, acually containerized?
